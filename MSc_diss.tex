\documentclass[11pt,a4paper]{article}
% Encoding and language
\usepackage[utf8]{inputenc}
\usepackage[english]{babel}
% Colors
\usepackage[usenames,dvipsnames]{color}
\usepackage{colortbl}
% Paper size and margins
%\usepackage{rotating}
\usepackage{pdflscape}
%\usepackage[a4paper,showframe]{geometry}
\usepackage{vmargin}
\setmarginsrb{2cm}{2cm}{2cm}{2cm}{0cm}{0cm}{0cm}{1.5cm}
\usepackage{pdflscape}
\usepackage{afterpage}
% Indenting
\usepackage{indentfirst}
%\setlength{\parindent}{1cm}
\setlength{\parskip}{0.5cm}
% Symbols
\usepackage{amsmath}
\usepackage{amsfonts}
\usepackage{amssymb}
% Equations
\usepackage{eulervm} % upright font in eqns (Palatino)
\usepackage{cool}
\usepackage{mathtools} % for arrows
\renewcommand{\theequation}{R\arabic{equation}}
% References
\usepackage{natbib}
\bibliographystyle{apalike}
% Hyperlinks and labels
%\usepackage{showkeys}
\usepackage[linktocpage=true,plainpages=false,pdfpagelabels=false]{hyperref}
\definecolor{citecolor}{rgb}{0.2,0.7,0.9}
\definecolor{urlcolor}{rgb}{0,0,1}
\hypersetup{
    colorlinks, linkcolor={Blue},
    citecolor={Blue}, urlcolor={Black}
}
% Figures
\usepackage{subcaption}
\usepackage{alphalph}
\usepackage{tikz}
\usetikzlibrary{shapes,arrows}
\usetikzlibrary{positioning}
\usetikzlibrary{calc}
% Tikz styles
\tikzstyle{reg} = [rounded rectangle, text width=2cm, fill=black!5, align=center, anchor=west,font=\sffamily\Large\bfseries, minimum height=1cm, inner sep=0]
\tikzstyle{rad} = [rounded rectangle, draw=none, text width=2cm, fill=red!10, align=center, anchor=west,font=\sffamily\Large\bfseries, minimum height=1cm, inner sep=0]
\tikzstyle{long} = [rounded rectangle, draw=none, text width=3.5cm, fill=black!5, align=center, anchor=west,font=\sffamily\Large\bfseries, minimum height=1cm, inner sep=0]   
\tikzstyle{add} = [scale=1.5, draw=none,fill=none,align=center]
\tikzstyle{emp} = [draw=none,fill=none]
\tikzstyle{ar} = [->,draw=black!70,line width=2]
\tikzstyle{ln} = [-,draw=black!70,line width=2]
% Title etc
\newcommand{\mytitle}{Investigation of the relationship between tropospheric ozone production efficiency and carbon bond emissions}
\newcommand{\authorname}{Maria Zamyatina}
\newcommand{\authornumber}{Student No: 100106685}
\newcommand{\degname}{Master of Science}
\newcommand{\univname}{University of East Anglia}
\newcommand{\univaddr}{University Plain \\ Norwich NR47TJ UK}
\newcommand{\depname}{School of Environmental Sciences}
\newcommand{\mydate}{\the\year}
%% Remove author and date from the maketitle command
%\makeatletter
%\renewcommand{\@maketitle}{
%\newpage
% \null
% \vskip 2em%
% \begin{center}%
%  {\LARGE \@title \par}%
% \end{center}%
% \par} \makeatother

% PDF meta-data
\hypersetup{pdftitle={\mytitle}}
\hypersetup{pdfauthor=\authorname}

\title{\mytitle}
% =========================================================================
\begin{document}

\input{MSc_diss_title}

\tableofcontents
\newpage

\begin{abstract}
Abstract.
\end{abstract}

\section{Introduction} \label{sec:intro}
The Earth’s atmosphere is the recipient of a vast range of chemical compounds emitted by natural and anthropogenic sources. The oxidation of these compounds is the main process taking place in the atmosphere that keeps their concentrations at levels that do not distort the chemical balance of the atmosphere \citep{Prinn2003}. In other words, oxidation is a removal “cleansing” mechanism for these compounds (various pollutants and greenhouse gases), and its rate is often referred to as the oxidizing capacity of the atmosphere. This capacity depends mainly on two factors: the global burden of the principal oxidants such as ozone ($O_3$), the hydroxyl radical ($OH$) and hydrogen peroxide ($H_2O_2$), and climate \citep{Prinn2003,Thompson1992}.

\subsection{Tropospheric ozone}
Tropospheric ozone ($O_3$) plays an important role in the chemical and thermal balance of the atmosphere. Although it accounts only for about 10\% of the total ozone, it is a primary precursor to the formation of hydroxyl radicals ($OH$) which in association with hydrogen peroxide ($H_2O_2$) determines the ‘cleansing’ capacity of the atmosphere \citep{Prinn2003,Tarasick2008,Thompson1992}. On the other hand, being a source of main pollutant removal agents, ozone is an air pollutant itself. Long exposure to its high concentrations causes respiratory problems in humans and damage to sensitive plant species \citep{Fowler2008}. Last but not least characteristic of tropospheric ozone is that it is an important greenhouse gas with a current estimated radiative forcing of  $0.40\pm 0.20~Wm^{–2}$ which is approximately one fifth of the $CO_2$ radiative forcing \citep{Hartmann2013,Myhre2013}. Due to these reasons a considerable interest in quantifying tropospheric ozone concentrations, budget and trends exists for more than a 100 years \citep{Becker2004}.

Unlike many other air pollutants, tropospheric ozone is not directly emitted, but produced following the oxidation of carbon monoxide ($CO$), methane ($CH_4$), and nonmethane volatile compounds (nmVOCs) in the presence of nitrogen oxides (NOx) \citep{Crutzen1973,Myhre2013}. The emissions of these precursors arise from both natural and anthropogenic sources which have intensified since pre-industrial times \citep{Parrish2014,Volz1988} (Figure \ref{fig:O3observations}). Presumably it has led to roughly a doubling in background tropospheric ozone concentration \citep{Guicherit2000,Hartmann2013,Tarasick2008,Vingarzan2004}, although contribution from another major source, transport from the stratosphere, may also took place \citep{Fowler2008}. Climate is also believed to be a key driver of ozone production and destruction since the chemical reactions affecting ozone concentration are strongly dependent on such climatic factors as temperature, rainfall and humidity. Therefore in a changing climate tropospheric ozone is not anymore a local air quality issue, but is a global pollution problem \citep{Fowler2008}.

\begin{figure}[h]
\includegraphics[width=\linewidth]{{./pics/Isaksen2009_O3}.png}
\caption{Observed surface ozone at different Northern Hemispheric surface stations \citep{Isaksen2009}.}
\label{fig:O3observations}
\end{figure}

According to the Intergovernmental Panel on Climate Change (IPCC) Fifth Assessment report (AR5) for present conditions (around year 2000) the global mean tropospheric ozone budget is approximately 331 Tg \citep{Myhre2013}. However, despite general agreement on how the drivers impact ozone concentrations at global and regional scales, its budget varies considerably between different modelling and observational studies \citep{Stevenson2006,Wild2007,Young2012}. Table \ref{tab:O3budget} demonstrates that the range of the global tropospheric ozone budget estimates is approximately 60 Tg which is 18\% of the mean. In terms of ozone production the variation is even higher and account for 28\% of the corresponding value. In that regard, two questions arise: why does the tropospheric ozone budget differ between models and observations? And secondly, why does it vary to this extent, especially in terms of production? The only way to answer these questions is to perform a rigorous investigation of the factors that drive tropospheric ozone and attribute those differences to respective factors \citep{Young2012}.

\begin{table}[h] % O3 budget
\centering
\caption{Summary of tropospheric ozone global budget model and observation estimates for present (about 2000) conditions. All uncertainties quoted as 1 standard deviation (68\% confidence interval). Adapted from \citep{Myhre2013}}
\label{tab:O3budget}
\begin{tabular}{ccl}
\hline
Burden, Tg & Production, Tg yr–1 & Reference \\
\hline
\multicolumn{3}{c}{Modelling studies} \\
$337\pm23$  & $4877\pm853$ & Young et al. (2013); ACCMIP \\
$323$	    & -	           & Archibald et al. (2011) \\
$330$	    & $4876$       & Kawase et al. (2011) \\
$312$       & $4289$       & Huijnen et al. (2010) \\
$334$       & $3826$       & Zeng et al. (2010) \\
$324$       & $4870$       & Wild and Palmer (2008) \\
$314$       & -	           & Zeng et al. (2008) \\
$319$	    & $4487$	   & Wu et al. (2007) \\
$372$	    & $5042$       & Horowitz (2006) \\
$349$	    & $4384$	   & Liao et al. (2006) \\
$344\pm39$	& $5110\pm606$ & Stevenson et al. (2006); ACCENT \\
$314\pm33$	& $4465\pm514$ & Wild (2007) (post-2000 studies) \\
\hline
\multicolumn{3}{c}{Observational studies} \\
$333$   	& -	           & Fortuin and Kelder (1998) \\
$327$	    & -	           & Logan (1999) \\
$325$	    & -	           & Ziemke et al. (2011); 60S–60N \\
$319–351$	& -	           & Osterman et al. (2008); 60S–60N \\
\hline
\end{tabular}
\end{table}

A considerable effort has been made to tackle these problems, and generally there are two possible ways of doing it. The first one implies the direct analysis of tropospheric chemical schemes used in climate-chemistry models. For example, Emmerson and Evans (2009) compared the state of the art Master Chemical Mechanism which contains approximately 5600 species and 13500 reactions \citep{Jenkin2002} with other six chemical schemes and found four significant variations between them. This approach is linked with big challenges since a modern global model is a very sophisticated representation of the climate-chemistry interactions, and therefore it is often difficult to separate influence of one reaction from another. However, even being so developed climate-chemistry models do not take into account all known chemical reactions taking place in the real troposphere. A complete representation requires many thousands of species and tens of thousands of reactions, which is beyond the numerical capabilities currently available. This significantly limits the practicable size of chemical schemes and typically involves a reduction of the number of VOCs considered and lumping the carbon from the discarded species into representative surrogates.

The second way to understanding the discrepancies between models and observations is to conduct an idealised theoretical study using a conceptual model. The advantage of this approach is that such kind of models can be quite easily constructed, and more importantly can be focused on representation of a certain chemical reactions which presumably could explain the discrepancy between models and observations. An example of such set of reactions is alkyl nitrate chemistry.

\subsection{Alkyl nitrates}
Alkyl nitrates are important tropospheric trace gases which are formed following the same chemistry that leads to the production of ozone \citep{Reeves2007}. The simplest example, in case of $CH_4$ oxidation, can be described by the set of reactions below (also shown in Figure 2).

%\documentclass[]{standalone}
\usepackage[utf8]{inputenc}
\usepackage[english]{babel}
\usepackage{amsmath}
\usepackage{amsfonts}
\usepackage{amssymb}

\usepackage{graphicx}
%\usepackage[left=2cm,right=2cm,top=2cm,bottom=2cm]{geometry}

% Figures
\usepackage{tikz}
\usetikzlibrary{shapes,arrows}
\usetikzlibrary{positioning}
\usetikzlibrary{calc}
%\usepackage{chemfig}

% Tikz styles
\tikzstyle{reg} = [rounded rectangle, text width=2cm, fill=black!5, align=center, anchor=west,font=\sffamily\Large\bfseries, minimum height=1cm, inner sep=0]
\tikzstyle{rad} = [rounded rectangle, draw=none, text width=2cm, fill=red!10, align=center, anchor=west,font=\sffamily\Large\bfseries, minimum height=1cm, inner sep=0]
\tikzstyle{long} = [rounded rectangle, draw=none, text width=3.5cm, fill=black!5, align=center, anchor=west,font=\sffamily\Large\bfseries, minimum height=1cm, inner sep=0]   
\tikzstyle{add} = [scale=1.5, draw=none,fill=none,align=center]
\tikzstyle{emp} = [draw=none,fill=none]
\tikzstyle{ar} = [->,draw=black!70,line width=2]
\tikzstyle{ln} = [-,draw=black!70,line width=2]


\begin{document}
\begin{tikzpicture}[node distance = 4cm, auto]

% Nodes: big cycle: CH4-CH302-CH3NO3-CH3O-HCHO-HO2-OH-O3
\node[reg] (ch4) {$CH_4$};
\node[add](foo1_1) at (3,-2) {};
\node[add](foo1_2) at (4,-2) {$O_2$\\$OH$};
\node[reg] (ch3o2) at (5,-4){$CH_3 O_2$};
\node[reg] (oh) at (5,4){$OH$};
\node[long] (ch3ooh) at (0,-8){$CH_3OOH$};
\node[reg] (oz1) at (0,8){$O_3$};
\node[emp] (foo2) at ($(oz1)!.5!(oh)$){};
\node[reg] (ho2) at (15,4){$HO_2$};
\node[add] (oz_up) at ($(ho2)!.5!(oh) + (0,0.5)$){$O_3$};
\node[reg] (h2o2) at (17.5,7.5){$H_2O_2$};
\node[add] (ho2_) at (16.5,6){$HO_2$};
\node[emp] (nox_top) at ($(ho2)!.5!(oh) + (0,1.5)$){};
\node[reg] (ch3o) at (15,-4){$CH_3 O$};
\node[reg] (hcho) at (20,0){$HCHO$};
\node[emp] (foo5) at (19,-2){};
\node[long] (ch3no3) at (10,-10){$CH_3 NO_3$};
\node[add] (foo1_2) at (7,-9) {$NO$};
%\node[emp] (dep1) at (11,-12){};
%\node[add] (dep1_) at (10.3,-11.3){deposition};

%Nodes: Bottom NOx cycle + misc
\node[add] (no_1) at (7,-6){$NO$};
\node[add] (no2_1) at (13,-6){$NO_2$};
\node[emp] (nox_bottom) at ($(no_1)!.5!(no2_1)+(0,2)$){};
\node[emp] (foo7) at (11.5,-7.5){};
\node[add] (hv_11) at (12,-7){$h\nu$};
\node[reg] (o3_1) at (9.25,-8.75){$O_3$};
\node[add] (o2_1) at (11.2,-7.9){$O_2$};
\node[emp] (foo3) at (16,-8){};
\node[add] (hv_12) at (15,-8.75){$h\nu$};
\node[emp] (foo4) at (19,-8){};
\node[add] (oh_) at (20,-6.5){$OH$};
\node[add] (o2_) at (18,-3){$O_2$};

%Nodes: Top NOx cycle + misc
\node[add] (no2_2) at ($(ho2)!.5!(oh)+(-3,3.5)$){$NO_2$};
\node[add] (no_2) at ($(ho2)!.5!(oh)+(3,3.5)$){$NO$};
\node[add] (hv_2) at (10,9){$h\nu$};
\node[reg] (o3_2) at (13,10.5){$O_3$};
\node[add] (o2_2) at (12.75,9.75){$O_2$};
\node[emp] (foo6) at (11.5,9.5){};

%Nodes: CO & CO2
\node[reg] (co) at ($(ho2)!.5!(oh)+(-4,-3)$){$CO$};
\node[reg] (co2) at ($(ho2)!.5!(oh)+(2,-3)$){$CO_2$};
\node[emp] (cc_o2) at ($(ho2)!.5!(oh)+(0,-2.5)$){};
\node[emp] (cc_1) at ($(ho2)!.5!(oh)+(-0.5,-1.5)$){};
\node[emp] (cc_2) at ($(ho2)!.5!(oh)+(0.5,-1.5)$){};
\node[add] (cc_o2) at ($(ho2)!.5!(oh)+(0,-2)$){$O_2$};

%Nodes: HCHO decompostion
\node[long] (co_2ho2) at (25,3){$CO~+~2HO_2$};
\node[add] (hcho_hv1) at ($(hcho)!.5!(co_2ho2)$){$h\nu$};
\node[long] (co_h2) at (25,0){$CO~+~H_2$};
\node[add] (hcho_hv2) at ($(hcho)!.5!(co_h2)+(0.25,0.25)$){$h\nu$};
\node[long] (co_ho2) at (25,-3){$CO~+~HO_2$};
\node[add] (hcho_oh) at ($(hcho)!.5!(co_ho2)+(0.25,0)$){$OH$};
%\node[long] (hno3dep) at (25,-6){\small $CO~+~HO_2~+~HNO_3$\\ deposition?};
%\node[add] (hcho_no3) at ($(hcho)!.5!(hno3dep)$){$NO_3$};

%Nodes: after decomposition HO_2 + CO_2
\node[emp] (decomp) at ($(co_2ho2)!.5!(co_ho2)+(3,0)$){};
\node[long] (decomp2) at ($(decomp)+(2,0)$){$HO_2~+~CO_2$};
\node[add] (decomp_oh) at ($(decomp)+(1,0.5)$){$OH$};

% Connections: big cycle
\draw[ln] (oh.west) to [out=180,in=90](foo1_1.center);
\draw[ln] (ch4.east) to [out=0,in=90](foo1_1.center);
\draw[ar] (foo1_1.center) to [out=270,in=180](ch3o2.west);
\draw[ar] (ch3o2) to (ch3ooh);
\node[add](foo1) at ($(ch3o2)!.5!(ch3ooh) + (-1,0)$) {$HO_2$};
%\draw[ar] (oz1) to node[right]{$h\nu$}(foo2.north west);
\draw[ar] (oz1) to (oh);
\node[add] (o3_to_OH) at ($(oz1)!.5!(oh) + (0.5,0.5)$){$h\nu$\\ $H_2O$};
%\draw[ar] (foo2.south east) to node[right]{$H_2O$}(oh);
\draw[ar] (ho2) to (h2o2);
\draw[ln] (ho2.north west) to [out=135,in=0](nox_top.center);
\draw[ar] (nox_top.center) to [out=180,in=45](oh.north east);
\draw[ar] (ho2) to (oh);
\draw[ln] (ch3o2) to (nox_bottom.center);
\draw[ar] (nox_bottom.center) to (ch3o);
\draw[ar] (ch3o2.south) to [out=270,in=180](ch3no3.west);
\draw[ln] (ch3o.east) to [out=0,in=270](foo5.center);
\draw[ar] (foo5.center) to [out=90,in=0](ho2.east);
\draw[ar] (foo5.center) to [out=90,in=180](hcho.west);

% Connections: bottom NOx cycle + misc
\draw[ln] (no_1.north) to [out=90,in=180](nox_bottom.center);
\draw[ar] (nox_bottom.center) to [out=0,in=90](no2_1.north);
\draw[ln] (no2_1.south) to [out=270,in=0](foo7.center);
\draw[ar] (foo7.center) to [out=180,in=90](o3_1.north);
\draw[ar] (foo7.center) to [out=180,in=270](no_1.south);
\draw[ln] (ch3no3.east) to [out=0,in=270](foo3.center);
\draw[ar] (foo3.center) to [out=90,in=270](ch3o.south);
\draw[ar] (foo3.center) to [out=90,in=0](no2_1.east);
\draw[ln] (ch3no3.east) to [out=0,in=270](foo4.center);
\draw[ar] (foo4.center) to [out=90,in=270](hcho.south);
\draw[ar] (foo4.center) to [out=90,in=0](no2_1.east);
%\draw[ar] (ch3no3.south) -- (dep1.center);

% Connections: top NOx cycle
\draw[ln] (no_2.south) to [out=270,in=0](nox_top.center);
\draw[ar] (nox_top.center) to [out=180,in=270](no2_2.south);
\draw[ln] (no2_2.north) to [out=90,in=180](foo6.center);
\draw[ar] (foo6.center) to [out=0,in=90](no_2.north);
\draw[ar] (foo6.center) to [out=0,in=270](o3_2.south);

%Connections: CO & CO2
\draw[ln] (oh.south east) to [out=315,in=180](cc_1.center);
\draw[ln] (co.north east) to [out=45,in=180](cc_1.center);
\draw[ln] (cc_1.center) to [out=0,in=180](cc_2.center);
\draw[ar] (cc_2.center) to [out=0,in=225](ho2.south west);
\draw[ar] (cc_2.center) to [out=0,in=135](co2.north west);

%Connections: HCHO decomp
\draw[ar] (hcho.east) to [out=0,in=180](co_2ho2.west);
\draw[ar] (hcho.east) to [out=0,in=180](co_h2.west);
\draw[ar] (hcho.east) to [out=0,in=180](co_ho2.west);
%\draw[ar] (hcho.east) to [out=0,in=180](hno3dep.west);

\draw[ln] (co_2ho2.east) to [out=0,in=180](decomp.center);
\draw[ln] (co_h2.east) to [out=0,in=180](decomp.center);
\draw[ln] (co_ho2.east) to [out=0,in=180](decomp.center);
\draw[ar] (decomp.center) to [out=0,in=180](decomp2.west);


\end{tikzpicture}

\end{document}

In the presence of UV light shorter than 320 nm and water vapour $O_3$ photolysis is the main source of OH radicals in the troposphere:
\begin{equation} \label{eq:O3hv}
O_3 + hv \rightarrow O(^1D) + O_2
\end{equation}
\begin{equation} \label{eq:O1D+H2O}
O(^1D) + H_2O \rightarrow 2OH
\end{equation}
Once formed the OH radicals react mainly with $CO$ and $CH_4$ to initiate the $O_3$ formation or removal cycles by producing peroxy radicals ($HO_2$, $CH_3O_2$):
\begin{equation} \label{eq:OH+CO}
OH + CO \rightarrow HO_2 + CO_2
\end{equation}
\begin{equation} \label{eq:OH+CH4}
OH + CH_4 \rightarrow CH_3O_2 + H_2O
\end{equation}

The fate of the peroxy radicals depends on the presence of NOx. At low NOx levels (remote regions), the $HO_2$ radical reacts with another HO2 to form hydrogen peroxide, H2O2, or with the methyl peroxy radical, $CH_3O_2$, to form methyl hydrogen peroxide ($CH_3OOH$):
\begin{equation} \label{eq:HO2+HO2}
HO_2 + HO_2 \rightarrow H_2O_2 + O_2
\end{equation}
\begin{equation} \label{eq:HO2+CH3O2}
HO_2 + CH_3O_2 \rightarrow CH_3OOH + O_2
\end{equation}
On a whole, at low NOx levels a net $O_3$ loss takes place, because the reaction sequence is initiated by $O_3$ photolysis. Plus under the same conditions, some additional $O_3$ removal occurs due to the re-action of $HO_2$ radicals with $O_3$,
\begin{equation} \label{eq:HO2+O3}
HO_2 + O_3 \rightarrow OH + 2O_2
\end{equation}
leading to the regeneration of OH radicals as part of an $O_3$ depleting HOx catalytic cycle.

At intermediate NOx levels (rural areas of most industrialised countries), the peroxy radicals react with nitrogen monoxide ($NO$) to form nitrogen dioxide ($NO_2$) with the subsequent photolysis of $NO_2$ generating $O_3$:
\begin{equation} \label{eq:HO2+NO}
HO_2 + NO \rightarrow NO_2 + OH
\end{equation}
\begin{equation} \label{eq:CH_3O_2+NO}
CH_3O_2 + NO \rightarrow NO_2 + CH_3O
\end{equation}
\begin{equation} \label{eq:NO2hv}
NO_2 + hv \rightarrow NO + O(^3P)
\end{equation}
\begin{equation} \label{eq:O3P+O2}
O(^3P) + O_2 + M \rightarrow O_3 + M
\end{equation}
As it is shown in Figure 2 reactions (8) and (9) form part of the free-radical propagated $O_3$ forming cycles, which may occur a number of times before being halted by a radical termination reactions (not shown). However, the reaction of $CH_3O_2$ with NO is a two-channel reaction \citep{Day2003}, and its second branch can terminate the $O_3$ forming cycle earlier by producing an alkyl nitrate, in this case methyl nitrate ($CH_3ONO_2$):
\begin{equation} \label{eq:CH3NO3form}
CH_3O_2 + NO \rightarrow CH_3ONO_2
\end{equation}

Being a reservoir for the short lived NOx and having atmospheric lifetimes ranging from several days to about a month \citep{Reeves2007} alkyl nitrates can be transported over long distances, and then lost through photolysis and reaction with $OH$:
\begin{equation} \label{eq:CH3NO3hv}
CH_3ONO_2 + hv \rightarrow CH_3O + NO_2
\end{equation}
\begin{equation} \label{eq:CH3NO3+OH}
CH_3ONO_2 + OH \rightarrow HCHO + NO_2
\end{equation}
It means that alkyl nitrates can ‘release’ $NO_2$ back to the atmosphere after a while, and in association with reactions (10) and (11) start a new $O_3$ formation cycle being quite away from the source.

The free-radical propagated oxidation of hydrocarbons leads to the generation of carbonyl compounds, in this case formaldehyde ($HCHO$). Its further oxidation and photolysis contributes to radical generation:
\begin{equation} \label{eq:HCHOhv1}
HCHO + hv \rightarrow 2HO_2 + CO
\end{equation}
\begin{equation} \label{eq:HCHOhv2}
HCHO + hv \rightarrow CO + H_2
\end{equation}
\begin{equation} \label{eq:HCHO+OH}
HCHO + OH \rightarrow HO_2 + CO
\end{equation}
As a result, the formation of $HCHO$ has an impact on the rates of the oxidation cycles described above, through secondary radical generation \citep{Fowler2008}.
At high NOx levels (urban environment) reactions (8) and (9) dominate for $CH_3O_2$ and $HO_2$, but $O_3$ formation becomes inhibited by further increases in NOx. In other words, at high NOx regime the $O_3$ formation rate becomes sensitive to the concentration of $CH_4$ or $CO$ and to inputs of other VOCs \citep{Fowler2008}.

To sum up, alkyl nitrates are produced as a minor branch of the reaction of peroxy radicals with NO \citep{Day2003}. Since these compounds are removed from the troposphere by photolysis and OH oxidation they can provide information on photochemical evolution of an air mass through relationships with their parent hydrocarbons \citep{Bertman1995}. Acting as reservoir species for reactive nitrogen (NOy) and having a relatively long atmospheric lifetimes alkyl nitrates play an important role in long range transport of NOx. Furthermore, alkyl nitrates link to ozone and carbonyl production making them useful tracers for both species \citep{Worton2010}. Despite their relative importance alkyl nitrates are usually not included in the climate-chemistry models, for example in UKCA (Reeves, personal communication), which potentially could lead to discrepancies between models and observations described before. 

!! from \citep{Finlayson-Pitts2000}
The relationship between the peroxy radical concentration and the ozone photolysis rate constant for these higher NO conditions can be again approximated using steady-state analysis [..]. While OH is recycled in its reactions with CO and CH4 via HO2, it is pernanetly removed at higher NOx concentrations by the reaction of OH with NO2, forming nitric acid:
OH + NO2 = M = HONO2 (113).
This reaction is primarily a daytime reaction because most OH sources are photolytic in nature. As a result, the NO2 reaction with OH competes!!!! with NO2 photolysis (page 290/993)!!
         Rate coeff (OH+NO2 = HNO3)  J(NO2 = NO+O)
Book     2.47e-30          7.00e-3
My model 1.15e-11          4.39e-03
Thus, the reaction with OH is not usually a dominant loss process for NO2, but it is still sufficiently fact to from significant amounts of HNO3 during the day, particularly in polluted regions with relatively large NO2 concentrations.
Nitric acid undergoes both wet and dry deposition rapidly and can be neutralized by ammonia, the major gaseous base found in the atmosphere.

\subsection{Motivation}
lalalaep

\section{Overall objective and specific aims}
\subsection{Overall objective}
The overall purpose of this research is to investigate the impact of alkyl nitrate chemistry on ozone production efficiency.
\subsection{Specific aims}
Specific aims are:
\begin{itemize}
\item to investigate of the relationship between ozone production and carbon bond emissions (see Data and Methods);
\item to analyse the impact of inclusion of alkyl nitrates chemistry on this relationship;
\item to analyse everything mentioned above in various NOx regimes.
\end{itemize}

\section{Literature review} \label{sec:data}
Alkyl nitrates ($RONO_2$) are organic tropospheric trace gases. They are produced photochemically through the oxidation of their parent hydrocarbons \citep{Roberts1990} or emitted directly from equatorial oceans \citep{Blake2003} and biomass burning \citep{Simpson2002}. Being involved in the ozone production chemistry, these species can act as reservoirs of nitrogen oxides which availability largely affects the production of ozone \citep{Reeves2007}. Since the alkyl nitrates have atmospheric lifetimes ranging from several days to about a month, they can redistribute nitrogen oxides presented over polluted continental regions to clean remote marine environments, and therefore are considered as one of the tracers for determining the anthropogenic influence \citep{Atherton1989, Reeves2007, Worton2005}.

Alkyl nitrates are removed from the atmosphere by photolysis \citep{Turberg1990} or through oxidation by hydroxyl radical ($OH$) \citep{Talukdar1997}. Photolysis is the dominant loss mechanism for the compounds with short carbon chains and becomes more important with decreasing carbon number. Reaction with $OH$ follows the opposite trend and is most important for the pentyl nitrates and higher.

Owing to the existence of the relationship between alkyl nitrates and their parent hydrocarbons, measurements of their concentrations can provide information on the extent of photochemical processing in the air mass. \cite{Bertman1995} developed a kinetic approach to describe this relationship through the evolution of the ratio of alkyl nitrate to its parent hydrocarbon as a function of time. This approach has been extensively used to study air mass photochemical ageing at several sites, in North America \citep{Bertman1995, Roberts1998}, China \citep{Simpson2006}, the British Isles \citep{Worton2010} and over the Pacific \citep{Simpson2003} and North Atlantic Ocean \citep{Reeves2007}.

\cite{Bertman1995} found that the theoretical ratios of  2-butyl nitrate to n-butane and 2- and 3-pentyl nitrates to pentane strongly agree with those measured from the ambient air showing no evidence for primary sources of these nitrates. On the contrary, measurements of the other ratios, ethyl/ethane, n-propyl nitrate/propane, and 2-propyl nitrate/propane, appeared to be significantly higher than predicted, relative to the 2-butyl nitrate/n-butane ratio, meaning that there should be an additional source of these alkyl nitrates. \cite{Roberts1998} generally approved the theory by comparing Bertman's conclusions with the results from a new observational dataset.

Using methyl nitrate as a tracer of marine $RONO_2$ \cite{Simpson2006} revealed that the South China Sea in not a region of strong $RONO_2$ emissions suggesting that photochemical rather marine production of alkyl nitrates is dominant at the site.

\cite{Worton2010} found that the photochemical source of the short-chain alkyl nitrates originates mostly from the photochemical oxidation and decomposition of longer-chain compounds rather than from the oxidation of the parent hydrocarbons as it is thought in the previous studies.

The brief description of the measurement sites and data is presented in section \ref{sec:data}. In section \ref{sec:method} the methodology used in this work is defined. Section \ref{sec:res} contains the results. Section \ref{sec:discuss} discusses the results, particularly focusing attention on the analysis of the discrepancy between the observational data and the results of theoretical modelling. Conclusions constitute section \ref{sec:conclusion}.

\section{Methodology} \label{sec:method}
\subsection{Model development and description}
A zero-dimensional (box) model is developed to investigate the impact of alkyl nitrate chemistry on ozone production efficiency. Its code is written using FACSIMILE for Windows. FACSIMILE is a special high-level programming language which is a powerful modelling tool designed to efficiently solve differential equations including those resulting from a set of chemical reactions.

The model is intended to represent the steady state of the troposphere under mean summer daytime meteorology and clear skies. To achieve that a number of assumptions have been made. First, the model assumes a uniform mixing of individual constituents of the troposphere, second, there are no atmospheric transport (no 'winds') and no emissions of species into the box meaning that the time evolution of each species is controlled by the chemical interactions alone and determined by their initial concentrations. The air temperature and relative humidity are assumed to be $25^{\circ}C$ and 70\% respectively and are fixed in all model runs. Other assumptions concerning the chemical mechanism can be found in the next section.

!! For long-lived species, averaging may be reasonable.

\subsection{Chemical mechanism}
The chemical mechanism of the model is based on recommendations from \citep{Atkinson2004} and information from the Master Chemical Mechanism, MCM v3.3 \citep{Jenkin1997,Saunders2003}, retrieved via website: http://mcm.leeds.ac.uk/MCM. On a whole it includes 90 species and 165 reactions. The mechanism is developed to represent the chemistry that governs the production and loss of ozone, hydrocarbons and alkyl nitrates. These include:
\begin{itemize}
\item the formation of ozone as a result of the photolysis of $NO_2$ followed by the reaction of ground-state oxygen atom ($O(^3P)$) with molecular oxygen and a third body, M ($N_2$ or $O_2$);
\item the loss of ozone through its photolysis and reactions with $O(^3P)$, $OH$, $HO_2$ and $NO$, among which the last is of particular interest because it gives back $NO_2$;
\item the formation of alkylperoxy radicals ($RO_2$) from oxidation of the hydrocarbons;
\item the competition of $NO$ and $HO_2$ for reaction with $HO_2$ and $RO_2$;
\item the formation of alkyl nitrates which adds up to the competition of $NO$ and $HO_2$ for reaction with the peroxy radicals;
\item the loss of alkyl nitrates through their photolysis and reaction with $OH$;
\item the subsequent reactions of hydroperoxides and aldehydes formed from the organic reactions mentioned above.
\end{itemize}
The full list of reactions can be found in Appendix ?? \ref{sec:appendix1}. It is important to mention that due to purely theoretical purpose of this research and its focus on investigation of the impact of alkyl nitrate chemistry on ozone production some of the reactions that lead to formation of species that serve as a sink for $NO_x$ are not included into the chemical mechanism. However, these reactions play an important role in the chemistry of the troposphere, and below here is a description of their role and anticipated effect of their exclusion.
\begin{itemize}
\item The formation of the nitric acid ($HNO_3$) is a major loss process for $NO_x$ during daytime in the atmosphere. Being a 'sticky' molecule $HNO_3$ readily adsorbs to surfaces, particularly if there is water on the surface, and deposits either in precipitation (wet deposition) or in dry form (dry deposition). Because of the high deposition velocity, dry deposition of $HNO_3$ can be responsible for much of the removal of inorganic nitrogen from the troposphere \citep{Finlayson-Pitts2000}, which means that exclusion of $HNO_3$ (formation and removal)/(chemistry) from the chemical mechanism potentially leads to overproduction of ozone in the model, especially at high $NO_2$ concentrations because the competition between $NO_2$ photolysis and its reaction with $OH$ radical ($NO_2 + OH \xrightarrow{M} HNO_3$) is ignored.
\item Since this research is focused on the daytime chemistry some reactions that are believed to be significant in nighttime are not included into the chemical mechanism of the model. Among them are reactions that involve the nitrate radical ($NO_3$) and dinitrogen pentoxide ($N_2O_5$).
\begin{itemize}
\item $NO_3$ radical rapidly photolyzes during the daylight hours, but in nighttime its concentration can increase to measurable levels \citep{Atkinson2008}. This increase in concentration and a lesser availability of the OH radical (which sources are mostly photolytic in nature) make $NO_3$ radical a major contributor to the night chemistry of organics in the troposphere \citep{Finlayson-Pitts2000}. However, exclusion of $NO_3$ radical reactions from our chemical mechanism should not have a huge impact on the results anyway since the mechanism considers mainly decomposition of alkanes with which $NO_3$ radical reacts to a generally minor extent \citep{Atkinson2008}???.
\item ! the nighttime NO3 radical reaction is of minor significance in term of the overall removal of the alkanes (Atkinson, 1990a)
\item In addition to reacting with organics, $NO_3$ also reacts with $NO_2$, forming dinitrogen pentoxide ($NO_3 + NO_2 \xleftrightarrow{M} N_2O_5$) which in turn is an important nighttime source of nitric acid through its rapid hydrolysis on wet surfaces and aerosol particles ($N_2O_5 + H_2O \xrightarrow{surf., aerosol} 2HNO_3$) \citep{Finlayson-Pitts2000}. Exclusion of $N_2O_5$ from the chemical mechanism assumes no cycling of nitrogen oxides within the troposphere through $NO_3$ and $N_2O_5$, eventually forming $HNO_3$, which are not included into the mechanism anyway.
\item ! In any case, because of this equilibrium, sink for N2O5 such as hydrolysis are, in essence, also sinks for NO3 as well.
\end{itemize}
\item Another species involved in $NO_x$ tropospheric chemistry is the nitrous acid ($HONO$). The major source of nitrous acid is believed to be heterogeneous 'dark' reactions of $NO_2$, including hydrolysis (reaction with water vapour) of $NO_2$ on aerosol and particulate matter surfaces in nighttime. During the daytime, $HONO$ can be formed by the reaction of $OH$ radical with $NO$ ($OH + NO \xrightarrow{M} HONO$). However, since $HONO$ strongly absorbs UV radiation, it rapidly photolyzes, and under particular conditions (when the nighttime $HONO$ concentration reaches almost 15 ppb \citep{Finlayson-Pitts2000}), photolysis of nighttime-generated $HONO$ can become a major source of $OH$ radicals in the early morning hours ($HONO + hv \rightarrow OH + NO$) \citep{Lammel1996}. Therefore exclusion of $HONO$ from the chemical mechanism potentially leads to underestimation of $OH$ concentrations, but having neither formation nor removal of $HONO$ should balance this effect out???.
\item Peroxyacetyl nitrate ($CH_3C(O)OONO_2$, PAN) is the most abundant organic nitrate in the troposphere \citep{Hewitt1994}. A classic precursor to the PAN is acetaldehyde ($CH_3CHO$) which oxidation produces PAN through the following reaction sequence:
\begin{itemize}
\item[] $CH_3CHO + OH \rightarrow CH_3CO + H_2O$
\item[] $CH_3CO + O_2 \xrightarrow{M} CH_3C(O)OO$
\item[] $CH_3C(O)OO + NO_2 \xrightarrow{M} CH_3C(O)OONO_2$.
\end{itemize}
Of all the possible fates for PAN in the atmosphere, thermal decomposition is usually the most important because the rate constant for PAN decomposition strongly depends on temperature. At low temperatures it is quite stable, but at higher temperatures it decomposes giving $NO_2$ and the acetylperoxy radical ($CH_3C(O)OO$) 
\begin{itemize}
\item[] $CH_3C(O)OONO_2 \xleftrightarrow{M} NO_2 + CH_3C(O)OO$.
\end{itemize}
This temperature dependence has important implications for the role of PAN. When it is formed at lower temperatures (e.g., at high altitude) or transported into colder regions, PAN can act as an $NO_x$ reservoir. Observations of PAN and the sum of reactive nitrogen ($NO_y$) confirm it revealing that PAN can constitute up to 90\% of the total $NO_y$ budget at higher northern latitudes or higher altitudes (e.g., \citep{Jacobi2000}). When an air mass containing PAN is transported into warmer regions, $NO_2$ tied up in PAN can be released back and cause the formation of $O_3$ and $HNO_3$ far away from the original $NO_x$ source. Moreover, if sufficient $NO$ is present, $CH_3C(O)OO$ released from PAN can react in the following manner:
\begin{itemize}
\item[] $CH_3C(O)OO + NO \rightarrow CH_3C(O)O + NO_2$
\item[] $CH_3C(O)O \rightarrow CH_3 + CO_2$
\end{itemize}
followed by the reaction of $CH_3$ with $O_2$, etc. to form $HCHO$ and $HO_2$, or $HCHO$, $CH_3OH$ and $CH_3OOH$ under low-$NO_x$ conditions. As a result of this chemistry, PAN can act as an accelerator for photochemical smog formation \citep{Finlayson-Pitts2000}. Exclusion of PAN from the chemical mechanism of the model means that we omit the global transport of $NO_x$ provided by PAN, and therefore do not fully estimate tropospheric ozone production. However, since the focus of this research is on the impact of alkyl nitrate chemistry on ozone such a miscount is acceptable due to purely theoretical purpose of the research.
\item Inclusion of all chemistry listed above into the model is a subject for future work.
\end{itemize}

The constructed chemical mechanism has a limited representation of organic chemistry. It includes (oxidation schemes)/(degradation reactions) of only seven alkanes: methane ($CH_4$), ethane ($C_2H_6$), propane ($C_3H_8$), n-butane ($nC_4H_{10}$), i-butane ($iC_4H_{10}$), n-pentane ($nC_5H_{12}$) and i-pentane ($iC_5H_{12}$). While the oxidation scheme of $CH_4$ is full, oxidation of other six alkanes is not full, but terminated at different oxidation steps.


It has been done due to computational restrictions and  because otherwise the mechanism would 
Another limitation of the model's chemical mechanism is that it does not consider all possible organic reactions (which otherwise could lead to a too large number of reactions but is focused on representing the oxidation of alkanes
full oxidation scheme for methane ($CH_4$) only .

!! from \citep{Finlayson-Pitts2000}
! In urban areas: air pollution is characterized more by the formation of ozone and other oxidants rather than by SO2, particles, and sulfuric acid. In these regions, the primary pollutants are NOx (mainly NO) and volatile organic compounds (VOC), which undergo photochemical reactions in sunlight to form a host of secondary pollutants, the most of which is O3. Some of these are criteria pollutants for which air quality standards have been set, such as O3, SO2, CO, NO2, PM10 and PM2.5. Other are so-called "trace" noncritetia pollutants, e.g., gaseous peroxyacetyl nitrate (CH3C(O)OONO2,,PAN), nitric acid (HNO3 or HONO2), formaldehyde (HCHO), and formic acid (HCOOH). The overall reaction is now written as:
VOC + NOx + hv = O3 + PAN + HNO3 + ... + Particles, etc. (6)

! Since the most alkoxy radicals ($RO$) react rapidly with $O_2$ under atmospheric conditions and most large (>=C4) alkoxy radicals also rapidly decompose or isomerize, alkyl nitrate formation from reaction RO+NO2 = RONO2 is relatively unimportant for most organics under atmospheric conditions while observations showed that the reaction RO2+NO = RONO2 gives a higher yield (from \citep{Atkinson2000}, actually from 1982).

! Since one of the main mechanisms of the alkyl nitrate production (>C3!!) is the gas gas phase oxidation of alkanes by OH in the presence of NOx (\citep{Newland2013}).

\subsection{Photolysis}
To estimate photolysis rates for a given geographical location, one must take into account the latitude and season, as well as the time of the day.

24-hour average photolysis rates are calculated externally (outside FACSIMILE program) for 1 July using the formula and parametrisations used in MCM v3.3. The formula that describes variation of photolysis rates with solar zenith angle for the core photolysis reactions was determined by the MCM's authors by using a two stream isotopic scattering model and by calculating rates for clear sky conditions at an altitude of 0.5 km on 1 July at a latitude of 45$^{\circ}$N.
has the following form:
\begin{equation} \label{eq:MCMphotolysis}
\renewcommand{\theequation}{\arabic{equation}}
J = l(cos\chi)^mexp(-nsec\chi)
\end{equation}
where $J$ - photolysis rate coefficient, $\chi$ - solar zenith angle, $l$, $m$ and $n$ - optimised parameters.
and it is a result of determination from the available data on absorption cross section and quantum data for the core reactions using two stream  Hayman.

The values of solar zenith angle are calculated using formulas from NOAA Solar Calculator (http://www.esrl.noaa.gov/gmd/grad/solcalc/).

\subsection{Design of experiments}
Hypothesis: ANs have an impact on O3.

To test the hypothesis box model runs were performed at a series of different NOx and VOCs initial concentrations, so that an $O_3$ isopleth plot could be constructed, similar to those found in \citep{Dodge1977} and \citep{Sillman1999}. The following initial concentrations were used:

\begin{table} % model setup: O3, CO
\caption{Fixed and variable parameters used in box model runs}
\centering
\begin{tabular}{ccc}
\hline
Parameter & Initial mixing ratio & Type \\
\hline
$H_2O$    & 2\%                  & Fixed    \\
$O_2$     & 21\%                 & Fixed    \\
$N_2$     & 78\%                 & Fixed    \\
$H_2$     & 525 ppb  ???         & Variable \\
$O_3$     & 40 ppb               & Variable \\
$CO$      & 100 ppb              & Variable \\
\hline
\end{tabular}
\end{table}	

\begin{table} % model setup: NOx
\caption{Model setup: NOx initial concentrations (ppb)}
\centering
\begin{tabular}{ccc}
\hline
$NO$      & $NO_2$      & NOx  ($=NO+NO_2$) \\
\hline
0.0015    & 0.0035      & 0.005 \\
0.007     & 0.018       & 0.025 \\
0.015     & 0.035       & 0.05  \\
0.03      & 0.07        & 0.1   \\
0.07      & 0.18        & 0.25  \\
0.15      & 0.35        & 0.5   \\
0.25      & 0.5         & 0.75  \\
0.3       & 0.7         & 1     \\
0.7       & 1.8         & 2.5   \\
1.5       & 3.5         & 5     \\
3         & 7           & 10    \\
\hline
\end{tabular}
\end{table}	

\begin{table} % model setup: VOCs
\caption{Model setup: VOCs initial concentrations (ppb)}
\centering
\begin{tabular}{ccccccccc}
\hline
Experiment number & $CH_4$ & $C_2H_6$ & $C_3H_8$ & $nC_4H_{10}$ & $iC_4H_{10}$ & $nC_5H_{12}$ & $iC_5H_{12}$ \\
\hline
1  & 1780 &	0.620 &	0.100 &	0.050 &	0.250 &	0.250 &	0.250 \\
2  & 1807 &	1.258 &	0.390 &	0.245 &	0.325 &	0.275 &	0.425 \\
3  & 1834 &	1.896 &	0.680 &	0.440 &	0.400 &	0.300 &	0.600 \\
4  & 1861 &	2.534 &	0.970 &	0.635 &	0.475 &	0.325 &	0.775 \\
5  & 1888 &	3.172 &	1.260 &	0.830 &	0.550 &	0.350 &	0.950 \\
6  & 1915 &	3.810 &	1.550 &	1.025 &	0.625 &	0.375 &	1.125 \\
7  & 1942 &	4.448 &	1.840 &	1.220 &	0.700 &	0.400 &	1.300 \\
8  & 1969 &	5.086 &	2.130 &	1.415 &	0.775 &	0.425 &	1.475 \\
9  & 1996 &	5.724 &	2.420 &	1.610 &	0.850 &	0.450 &	1.650 \\
10 & 2023 &	6.362 &	2.710 &	1.805 &	0.925 &	0.475 &	1.825 \\
11 & 2050 &	7.000 &	3.000 &	2.000 &	1.000 &	0.500 &	2.000 \\
\hline
\end{tabular}
\end{table}

\subsection{Bertman}
Below is the simplified scheme of atmospheric hydrocarbon oxidation leading to the formation of alkyl nitrates (see also Fig. \ref{fig:scheme_bertman}). It begins from the process of $H$ abstraction by the $OH$ radical where the fraction of the primary or secondary radical is denoted by $\alpha_1$.
\begin{equation} \label{eq:alkane_oh}
RH + \bullet OH \xrightarrow{k_1, \alpha_1} R\bullet + H_2O
\end{equation}
The resulting radical rapidly reacts with $O_2$ and forms an alkyl peroxy radical ($R\bullet$):
\begin{equation} \label{eq:rad_o2}
R\bullet + O_2 \xrightarrow{k_2} ROO\bullet
\end{equation}
Then the alkyl peroxy radical reacts with $NO$, when it can either lose an oxygen atom to form alkoxy radical ($RO\bullet$) or bond with $NO$ to form an alkyl nitrate. The former reaction is a crucial in the net production of tropospheric ozone, since photolysis of $NO_2$ is the main photochemical source of $O_3$ in the troposphere.
\begin{subequations} \label{eq:peroxy_no0}
\begin{align}
ROO\bullet + NO &\xrightarrow{k_3, (1-\alpha_3)} RO\bullet + NO_2 \label{eq:peroxy_no1}\\
ROO\bullet + NO &\xrightarrow{k_3, \alpha_3} RONO_2 \label{eq:peroxy_no2}
\end{align}
\end{subequations}
In other words, a branching ratio $\alpha_3$ of the reactions between alkyl peroxy radicals and $NO$ leads to formation of nitrate products via rearrangement of a high-energy intermediate \citep{Bertman1995}. The branching ratio $\alpha_3$ is larger when the parent alkane is larger and is a function of temperature and pressure.

\begin{figure}[h]
\centering
\begin{tikzpicture}[node distance = 4cm, auto]
\node[reg] (rh) {$RH$};
\node[rad] (roo) at (4,0){$ROO\bullet$};
\node[add] (oh) at (3,0.3) {$\bullet OH$};
\node[rad] (roor) at (4,-2){$ROOR'$};
\node[add] (r_oo) at (3.7,-1) {$R'OO\bullet$};
\node[reg] (rono2) at (2,3){$RONO_2$};
\node[rad] (ro) at (6,3){$RO\bullet$};
\node[add] (alpha) at ($(roo)!.5!(rono2) + (-1,-0.2)$) {$\alpha_3$};
\node[add] (alpha2) at ($(roo)!.5!(ro) + (1,-0.2)$) {$1-\alpha_3$};
\node[add] (no) at ($(roo)+(0,1.5)$) {$NO$};
\node[reg] (rcho) at (9,3){$RCHO$};
\node[add] (prod) at (0,3){\textit{products}};

\draw[ar] (rh.east) to [out=0,in=180](roo.west);
\draw[ar] (roo.south) to [out=270,in=90](roor.north);
\draw[ar] (roo.north) to [out=90,in=270](rono2.south);
\draw[ar] (roo.north) to [out=90,in=270](ro.south);
\draw[ar] (rono2.west) to [out=180,in=0](prod.east);
\draw[ar] (ro.east) to [out=0,in=180](rcho.west);
\end{tikzpicture}
\caption{Schematic representation of the alkyl nitrate chemistry and their formation through the photooxydation of alkanes. Adapted from \citep{Bertman1995}.}
\label{fig:scheme_bertman}
\end{figure}

Apart from $NO$, alkyl peroxy radicals can react with each other and give peroxides, without formation of nitrates. As the eqs. \eqref{eq:peroxy_no0} outline, whether $ROO\bullet$ will bond with nitrates or with other peroxy radicals depends partially on $NO$ concentration. The reaction between $ROO\bullet$ and $NO$ is more likely when nitrate concentration is high.

The alkyl nitrates are removed from the atmosphere through photolysis and reaction with hydroxyl radical.
\begin{equation} \label{eq:an_photo}
RONO_2 + h\nu \xrightarrow{j_4} RO\bullet + NO_2
\end{equation}
\begin{equation} \label{eq:an_oh}
RONO_2 + OH \xrightarrow{k_5} \mathit{products}
\end{equation}
The relative importance of these two loss reactions depends generally on the net UV intensity and the size of the carbon chain in alkyl nitrates. Photolysis is relatively more important for nitrates with $< C_4$, while nitrates with longer chains ($\geq C_5$) tend to react faster with $OH$. In case of butyl nitrates both mechanisms are of similar significance \citep{Worton2010}.

Assuming that the only source of alkyl nitrates is the $OH$ reaction with alkane, and that reactions take place in $NO$-rich atmosphere (so no $ROO\bullet$ self-reactions occur), the alkyl nitrate part in this generalised scheme can be reduced to the two equations of formation and loss:
\begin{subequations} \label{eq:an_form_loss0}
\begin{align}
RH &\xrightarrow{\beta k_A} RONO2 \label{eq:an_form_loss1}\\
RONO_2 &\xrightarrow{k_B} \mathit{products} \label{eq:an_form_loss2}
\end{align}
\end{subequations}
where $k_A = k_1[OH]$ and $k_B=j_4 + k_5[OH]$. The factor $\beta=\alpha_1\alpha_3$ takes into account the fraction of $H$ atom abstraction from the particular hydrocarbon (with the reaction rate $k_1$) and the branching ratios leading to the nitrate. The photolytic loss rate of $RONO_2$ is denoted by $j_4$, and $k_5$ is the rate of alkyl nitrate destruction by $OH$.

These kinetic equations lead to the ordinary differential equations which can be integrated in time ($t$) giving a relationship between the alkyl nitrates and their parent alkanes concentrations:
\begin{equation} \label{eq:integral}
\frac{[RONO_2]}{[RH]} = \frac{\beta k_A}{k_B - k_A}\left(1-e^{(k_A - k_B)t}\right)+\frac{[RONO_2]_0}{[RH]_0}e^{(k_A - k_B)t}
\end{equation}
where subscript $0$ denotes the initial concentrations.

A further good approximation is that there is no alkyl nitrate initially (\textit{i.e.} no direct emissions), hence, the second term in eq. (\eqref{eq:integral}) becomes zero. Finally, the evolution of alkyl nitrates in an air mass can be modelled using the following formula.
\begin{equation} \label{eq:model}
\frac{[RONO_2]}{[RH]} = \frac{\beta k_A}{k_B - k_A}\left(1-e^{(k_A - k_B)t}\right)
\end{equation}
It has to be noted that eq. \eqref{eq:model} assumes no mixing with surrounding air. This simplification is reasonable only when mixing affects both the parent alkane and the corresponding alkyl nitrate similarly, keeping the ratios in eqs. \eqref{eq:integral} and \eqref{eq:model} the same \citep{Reeves2007}.

\section{Results} \label{sec:res}
Dependence of alkyl nitrate yield on carbon number (paper about n-alkanes): bigger yield of ANs from larger carbon number means that the potential for contributing to photochemical air pollution may be less for the larger (C>6) n-alkanes that for the smaller ones.

\subsection{Justification of model performance}
!! from \citep{Finlayson-Pitts2000}
Urban:
Certain reproducible features of time-concentration profiles for  pollutants are observed in "smoggy" ambient air. Figure 1.3, a classic example of historical interest, shows such profiles for NO, NO2, and total oxidant (mainly O3) in Pasadena, California, during a  severe photochemical air pollution episode in July 1973.
Reproducible  features  include  the  following:
\begin{itemize}
\item In the early morning, the concentration of NO rises and  reaches a maximum at a time that approximately coincides with the  maximum emissions of NO, in this case, peak  automobile  traffic; 
\item Subsequently, NO2 rises to a maximum;
\item Oxidant (e.g., O3) levels, which are relatively low in the  early morning, increase significantly about noon when the NO concentration drops to a low value.
\end{itemize}

\section{Discussion} \label{sec:discuss}
The experimental results presented in the previous section provide an opportunity to speculate about the behaviour of our simple model of alkyl nitrate production and loss with time in comparison with the observational data. 

\subsection{Short-chained alkyl nitrates} \label{sec:short_an}
\subsubsection{Short photochemical processing times} \label{sec:short_an_short_time}

\section{Conclusion} \label{sec:conclusion}


\newpage
\clearpage
\thispagestyle{empty}% empty page style (?)
\begin{landscape}% Landscape page
\centering
	%\input{subfig_28_land}
\end{landscape}

\bibliography{diss_refs}

\section{Appendix I} \label{sec:appendix1}

\end{document}
